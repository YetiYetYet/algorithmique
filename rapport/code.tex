\section{Code}
\begin{verbatim}
#include "stdio.h"
#include "stdlib.h"
#include "string.h"
#include "assert.h"
#include "limits.h"
#include "time.h"

// Pense Bete : Partir de id-1

typedef struct station {
	int id;
	char nomStation[50];
	char ligne[15];
	struct station ** correspondance;
	int nbCorrespondance;
} station;

typedef struct metro
{
	station * stations;
	int nbstations;
} metro;

typedef struct dijk {
	char dejaVu;
	char ligneActuel;
	int stationAvant;
	int poids;
} dijk;


typedef struct entre{
	int a;
	int b;
} entre;

typedef struct Bstation{
	char nomStation[50];
	char ligne[15];
	int premierBrin;
} Bstation;

typedef struct brin{
	int indexStation;
	int brinSuivant;
} brin;

typedef struct Bmetro {
	Bstation * Bstations;
	int nbStations;
	brin * brins;
	int nbBrins;
} Bmetro;

metro creerMetro(){
	int a = 1;

	// Declaration metro
	metro metroBerlin;
	metroBerlin.nbstations = 173;
	metroBerlin.stations = malloc(sizeof(station)*173);
	assert(metroBerlin.stations);

	// Declaration Station

	station *station1 = metroBerlin.stations;
	station1->id = 1;
	strcpy(station1->nomStation, "Uhlandstraße");
	strcpy(station1->ligne, "U1");
	station1->nbCorrespondance=1;
	station1->correspondance = malloc(sizeof(station*)*station1->nbCorrespondance);
	assert(station1->correspondance);

	station *station2 = metroBerlin.stations+a;
	station2->id = 2;
	strcpy(station2->nomStation, "Kurfürsten-damm");
	strcpy(station2->ligne, "U1 U9");
	station2->nbCorrespondance=4;
	a++;
	station2->correspondance = malloc(sizeof(station*)*station2->nbCorrespondance);
	assert(station2->correspondance);

	station *station3 = metroBerlin.stations+a;
	station3->id = 3;
	strcpy(station3->nomStation, "Wittenberplatz");
	strcpy(station3->ligne, "U1 U2 U3");
	station3->nbCorrespondance=4;
	a++;
	station3->correspondance = malloc(sizeof(station*)*station3->nbCorrespondance);
	assert(station3->correspondance);

	
	[...]


	station *station173 = metroBerlin.stations+a;
	station173->id = 173;
	strcpy(station173->nomStation, "Spittelmarkt");
	strcpy(station173->ligne, "U2");
	station173->nbCorrespondance=2;
	a++;
	station173->correspondance = malloc(sizeof(station*)*station173->nbCorrespondance);
	assert(station173->correspondance);



	// Correspondance Ligne 1

	*(station1->correspondance) = station2;

	*(station2->correspondance) = station1;
	*(station2->correspondance+1) = station3;
	*(station2->correspondance+2) = station23;
	*(station2->correspondance+3) = station50;

	*(station3->correspondance) = station23;
	*(station3->correspondance+1) = station4;
	*(station3->correspondance+2) = station2;
	*(station3->correspondance+3) = station51;

	[...]

	*(station172->correspondance) = station87;
	*(station172->correspondance+1) = station147;
	return metroBerlin;
}


Bmetro creerMetroBrin(metro m){
	int i=0;
	int j=0;
	int b;
	int c;
	//int occ;
	int temp=INT_MAX;
	int temp2=INT_MAX;
	//maloc du nombre de correspondance*2+1 pour l'index 0 ?
	//Declaration metro
	Bmetro BmetroBerlin;
	BmetroBerlin.nbStations = m.nbstations;
	BmetroBerlin.Bstations = malloc(sizeof(Bstation)*(BmetroBerlin.nbStations));
	BmetroBerlin.nbBrins = 1;
	for(i=0; i<m.nbstations; i++){
		BmetroBerlin.nbBrins+=((m.stations+i)->nbCorrespondance);
	}
	BmetroBerlin.brins = malloc(sizeof(Bstation)*BmetroBerlin.nbBrins);
	assert(BmetroBerlin.Bstations);
	for(i=0; i<m.nbstations;i++){
		strcpy((BmetroBerlin.Bstations+i)->nomStation, (m.stations+i)->nomStation);
		strcpy((BmetroBerlin.Bstations+i)->ligne, (m.stations+i)->ligne);
	}
	
	//Initialisation Brin
	for(i=0; i<BmetroBerlin.nbBrins;i++){
		(BmetroBerlin.brins+i)->brinSuivant=INT_MAX;
		(BmetroBerlin.brins+i)->indexStation=INT_MAX;
	}
	for(i=0; i<BmetroBerlin.nbStations;i++){
		(BmetroBerlin.Bstations+i)->premierBrin=INT_MAX;
	}
	b=1;
	for(i=0; i<BmetroBerlin.nbStations;i++){
		for(j=0; j<(m.stations+i)->nbCorrespondance; j++){
			c=(*((m.stations+i)->correspondance+j))->id-1;
			if(i < c){
				if((BmetroBerlin.Bstations+c)->premierBrin==INT_MAX)
					(BmetroBerlin.Bstations+c)->premierBrin=BmetroBerlin.nbBrins/2-b;
				if((BmetroBerlin.Bstations+i)->premierBrin==INT_MAX)
					(BmetroBerlin.Bstations+i)->premierBrin=BmetroBerlin.nbBrins/2+b;
				(BmetroBerlin.brins+(BmetroBerlin.nbBrins/2+b))->indexStation=i;
				(BmetroBerlin.brins+(BmetroBerlin.nbBrins/2-b))->indexStation=c;
				b++;
			}
			//printf("%d\n", b);
		}
	}
	// Debug :
	// for(i=0; i<BmetroBerlin.nbBrins; i++){
	// 	if(i > 185)
	// 		printf("Brin %d(%d)-> %d Brin Suivant :\n", i, i-BmetroBerlin.nbBrins/2,(BmetroBerlin.brins+i)->indexStation);
	// 	if(i == 185)
	// 		printf("Brin %d(0) -> %d Brin Suivant :\n", i, (BmetroBerlin.brins+i)->indexStation);
	// 	if(i < 185)
	// 		printf("Brin %d(%d)-> %d Brin Suivant :\n", i, i-(BmetroBerlin.nbBrins/2),(BmetroBerlin.brins+i)->indexStation);
	// }
	for(i=0; i<BmetroBerlin.nbStations; i++){
			for(j=0;j<BmetroBerlin.nbBrins; j++){
				//printf("%d\n", j);
				if((BmetroBerlin.brins+j)->indexStation == i){
					temp = j;
					temp2 = j;
					(BmetroBerlin.brins+j)->brinSuivant = temp;
					j++;
					break;
				}
			}
			while(j<BmetroBerlin.nbBrins){
				if((BmetroBerlin.brins+j)->indexStation == i){
					(BmetroBerlin.brins+j)->brinSuivant = temp2;
					temp2 = j;
				}
				j++;
			}
			(BmetroBerlin.brins+temp)->brinSuivant =temp2;
	}
	printf("\n\n\n");

	// debug
	// for(i=0; i<BmetroBerlin.nbBrins; i++){
	// 	if(i > 185)
	// 		printf("Brin %d(%d)-> %d Brin Suivant :%d\n", i, i-BmetroBerlin.nbBrins/2,(BmetroBerlin.brins+i)->indexStation, (BmetroBerlin.brins+i)->brinSuivant);
	// 	if(i == 185)
	// 		printf("Brin %d(0) -> %d Brin Suivant :%d\n", i, (BmetroBerlin.brins+i)->indexStation, (BmetroBerlin.brins+i)->brinSuivant);
	// 	if(i < 185)
	// 		printf("Brin %d(%d)-> %d Brin Suivant : %d\n", i, i-(BmetroBerlin.nbBrins/2),(BmetroBerlin.brins+i)->indexStation, (BmetroBerlin.brins+i)->brinSuivant);
	// }
	return BmetroBerlin;
}


void afficheStation(metro m){
	int i;
	for(i=0; i < m.nbstations; i++){
		printf("La station %d se nomme : \033[4m%s\033[0m disponible sur la/les ligne(s) : %s.\n",i+1,(m.stations+i)->nomStation, (m.stations+i)->ligne);
	}
}

void afficheStation2(Bmetro m){
	int i;
	for(i=0; i < m.nbStations; i++){
		printf("La station %d se nomme : \033[4m%s\033[0m disponible sur la/les ligne(s) : %s.\n",i+1,(m.Bstations+i)->nomStation, (m.Bstations+i)->ligne);
	}
}

char memeLigne2(metro m, char ligneActuel, int idb){
	int i;

	for(i=1; i < 11;i=i+3){
		//printf("%c =? %c\n", temp1[i], temp2[j]);
		if((m.stations+idb)->ligne[i] == ligneActuel && (m.stations+idb)->ligne[i-1] == 'U')
			return 1;
	}
	return 0;
}

char BmemeLigne(Bmetro m, char ligneActuel, int idb){
	int i;

	for(i=1; i < 11;i=i+3){
		//printf("%c =? %c\n", temp1[i], temp2[j]);
		if((m.Bstations+idb)->ligne[i] == ligneActuel && (m.Bstations+idb)->ligne[i-1] == 'U')
			return 1;
	}
	return 0;
}

char ligneComp(metro m, int ida, int idb){
	//printf("Appel ligneComp\n");
	int i;
	int j;

	for(j=1; j < 11;j=j+3){
		for(i=1; i < 11;i=i+3){
			if((m.stations+ida)->ligne[i] == (m.stations+idb)->ligne[j]){
				//printf("Insertion boucle avec : %c\n", (m.stations+ida)->ligne[i]);
				return (m.stations+ida)->ligne[i];
			}
		}
	}
	return 0;
}

char BligneComp(Bmetro m, int ida, int idb){
	//printf("Appel ligneComp\n");
	int i;
	int j;

	for(j=1; j < 11;j=j+3){
		for(i=1; i < 11;i=i+3){
			if((m.Bstations+ida)->ligne[i] == (m.Bstations+idb)->ligne[j]){
				//printf("Insertion boucle avec : %c\n", (m.stations+ida)->ligne[i]);
				return (m.Bstations+ida)->ligne[i];
			}
		}
	}
	return 0;
}

int oppose(Bmetro m, int brin){
	int temp = brin;
	if(brin < m.nbBrins){
		temp = (m.nbBrins/2)-brin;
		brin = (m.nbBrins/2)+temp;
		return brin;
	}
	else{
		brin = (m.nbBrins)-brin;
		return brin;
	}

}

void dijkstraPurSansModificationJulie(metro m, int ida, int idb){
	int securite;
	int i;
	int b;
	int debut = ida;
	int poids;
	dijk liste[m.nbstations];

	// Initiation Dijkstra
	for(i=0;i<m.nbstations;i++){
		liste[i].dejaVu = 0;
		liste[i].poids = INT_MAX;
		liste[i].stationAvant = -1;
	}
	liste[ida].poids = 0;
	liste[ida].ligneActuel = 0;

	//Dijkstra
	while(ida != idb){
		liste[ida].dejaVu = 1;
		
		// Traitement des correspondances.
		for(i=0; i < (m.stations+ida)->nbCorrespondance; i++){
			b= ((*((m.stations+ida)->correspondance+i))->id-1);
			if(liste[b].dejaVu !=1){
				// Erreur Potentiel, id-1 pour l'index 0 ? Trop de confusion. Note a moi même : faire le fucking début de graphes a 0 pour eviter la confusion.
				if(memeLigne2(m, liste[ida].ligneActuel, b)){
					poids = liste[ida].poids+1;
				}
				else{
					poids = liste[ida].poids+2;
				}
				if(poids < liste[b].poids){
					if(memeLigne2(m, liste[ida].ligneActuel, b))
						liste[b].ligneActuel = liste[ida].ligneActuel;
					else
						liste[b].ligneActuel = ligneComp(m, ida, b);
					liste[b].poids = poids;
					liste[b].stationAvant = (m.stations+ida)->id-1;
				}
			}
		}
		securite = ida;
		poids = INT_MAX;
		for(i=0; i<m.nbstations;i++){

			if(!liste[i].dejaVu && liste[i].poids<poids){
				poids = liste[i].poids;
				ida = i;
			}
		}	
		if(ida == securite){
			printf("%d\n", ida);
			printf("La station %s n'as pas pus être atteinte, allez y à pied.\n", (m.stations+idb)->nomStation);
			return;
		}
	}
	// Affichage de bas en haut
	i = idb;
	printf("Attention, Le trajet se lit de bas en haut !\n\n");
	while(i!=debut){
		printf("%s via la ligne U%c\n", (m.stations+i)->nomStation, liste[i].ligneActuel);
		i = liste[i].stationAvant;
	}
	printf("%s\n", (m.stations+debut)->nomStation);
}


void DijkstraBrin(Bmetro m, int ida, int idb){
	int securite;
	int i;
	int b;
	int debut = ida;
	int poids;
	int temp;
	int c;
	int temp2;
	int brinop;
	dijk liste[m.nbStations];

	// Initiation Dijkstra
	for(i=0;i<m.nbStations;i++){
		liste[i].dejaVu = 0;
		liste[i].poids = INT_MAX;
		liste[i].stationAvant = -1;
	}
	liste[ida].poids = 0;
	liste[ida].ligneActuel = 0;

	//Dijkstra
	while(ida != idb){
		liste[ida].dejaVu = 1;
		temp = INT_MAX;
		c = 0;

		temp2 = (m.brins+((m.Bstations+ida)->premierBrin))->brinSuivant;
		brinop = oppose(m, (m.Bstations+ida)->premierBrin);
		b = (m.brins+brinop)->indexStation;
		while(temp != (m.Bstations+ida)->premierBrin){
			if(c){
					brinop = oppose(m, temp);
					b = (m.brins+brinop)->indexStation;
			}else{

				c++;
			}
			temp = temp2;
			temp2 = (m.brins+temp)->brinSuivant;
			if(liste[b].dejaVu != 1){
				if(BmemeLigne(m, liste[ida].ligneActuel, b))
					poids = liste[ida].poids+1;
			}
			else{
				poids = liste[ida].poids+2;
			}
			if(poids < liste[b].poids){
				if(BmemeLigne(m, liste[ida].ligneActuel, b))
						liste[b].ligneActuel = liste[ida].ligneActuel;
				else
					liste[b].ligneActuel = BligneComp(m, ida, b);
				liste[b].poids = poids;
				liste[b].stationAvant = ida;
			}
		}
		securite = ida;
		poids = INT_MAX;
		for(i=0; i<m.nbStations;i++){

			if(!liste[i].dejaVu && liste[i].poids<poids){
				poids = liste[i].poids;
				ida = i;
			}
		}	
		if(ida == securite){
			printf("%d\n", ida);
			printf("La station %s n'as pas pus être atteinte, allez y à pied.\n", (m.Bstations+idb)->nomStation);
			return;
		}
	}
	// Affiche de bas en haut
	printf("Attention, Le trajet se lit de bas en haut !\n\n");
	i = idb;
	while(i!=debut){
		printf("%s via la ligne U%c\n", (m.Bstations+i)->nomStation, liste[i].ligneActuel);
		i = liste[i].stationAvant;
	}
	printf("%s\n", (m.Bstations+debut)->nomStation);
}


entre recupStation(){
	entre ut;
	ut.a = 0;
	ut.b = 0;
		printf("Entrer la station de départ :\n");
		scanf("%d", &ut.a);

		printf("Entrer la station d'arrivé :\n");
		scanf("%d", &ut.b);
	return ut;
}

int main(){
	clock_t debut;
	debut = clock();
	metro metroBerlin = creerMetro();
	Bmetro BmetroBerlin = creerMetroBrin(metroBerlin);
	printf("\nTemps : %f\n", (double)(clock () - debut) / CLOCKS_PER_SEC);
	afficheStation(metroBerlin);
	entre ut = recupStation();
	//afficheStation2(BmetroBerlin);
	debut = clock();
	printf("\n\nStructure Vecteur :\n\n");
	dijkstraPurSansModificationJulie(metroBerlin, (ut.a-1), (ut.b-1));
	printf("\n\nStructure Brin :\n\n");
	DijkstraBrin(BmetroBerlin, (ut.a-1), (ut.b-1));
	printf("\nTemps : %f\n", (double)(clock () - debut) / CLOCKS_PER_SEC);
}

\end{verbatim}