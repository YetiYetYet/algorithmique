\section{Le code et moi}
\subsection{Les difficultés}
Dans ce projet, j'ai exprimé beaucoup de difficulté dans beaucoup de domaine. Mes plus grosse furent de comprendre les brins et d'adaptés le précédent graphes en brins (Pour moi les brins sont encore moins intuitifs que les regex). J'ai perdu beaucoup de temps en oubliant a chaque fois que le vecteurs des brins partait de 0  et non pas de -quelquechose, réalisé le graphes fut aussi dur pour moi. J'ai mis beaucoup de temps avant de trouver un moyen de le remplir. En ce qui concerne Dijkstra, l'algorithme en soit n'est pas très compliqué et ne néccessite pas beaucoup de connaissances en mathématiques pour le comprendre mais cette fois ci ce sont beaucoup d'erreurs d'étourderie qui mon fait perdre beaucoup de temps et d'espoirs. 

\subsection{Les améliorations possibles}
Il existe beaucoup beaucouo d'améliorations possible dans mon code. La première serait de creer le graphes autrement qu'en brut nottament avec un fichier texte ou une base de donnée déja disponible sur les data de googles. La secondes (plus esthétique) serait de divisé le programme en plusieurs fichier. Au niveau code, le plus important serait d'améliorer le choix des lignes communes, en effets, mon programme ne se contente que de prendre la premières ligne communes entre les deux stations et ne compare pas la meilleur à prendre. Au niveau optimisation, il y a moyen d'améliorer les ressources en mémoire et temps à utiliser. Nottament en évitant d'appeller plusieurs fois les mêmes fonctions et en stockant les résultat ou encore en changeant les petites valeurs de certaines variables par des char.

